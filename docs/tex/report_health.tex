\documentclass[english, a4paper, 11pt]{article}
\usepackage{graphicx}
\usepackage{float}
\usepackage{microtype}
\usepackage[T1]{fontenc}
\usepackage[utf8]{inputenc}
\usepackage{csquotes}
\usepackage{babel}

\usepackage[bookmarks=true,bookmarksnumbered]{hyperref}
\hypersetup{
	pdftitle={SSD26 - Medical Records},
	pdfauthor={BOTTON David, PEETROONS Simon, GERDAY Léandre, VARGA Ferenc, ANDRIANIRINA Mino},
	pdfkeywords={he2b, ulb, master of cybersecurity, medical records, E2EE, secure software design},
	bookmarksnumbered,
    breaklinks=true,
    colorlinks=true,
    linkcolor=blue,
    citecolor=black,
    urlcolor=blue,
}

\usepackage{lmodern}
\usepackage{amsmath,amssymb,textcomp}
\usepackage[style=ieee]{biblatex}
\bibliography{references}

\usepackage{booktabs,caption}
\captionsetup[table]{skip=1ex}

\usepackage{tikz}
\graphicspath{{figures/}}

\usepackage{tocloft}
\renewcommand{\cftsecleader}{\cftdotfill{\cftdotsep}}
\renewcommand{\cftsecfont}{\bfseries\large}
\renewcommand{\cftsecpagefont}{\bfseries\large}
\cftsetindents{section}{1.5em}{3.5em}
\setlength{\cftbeforesecskip}{8pt}

\usepackage[
headheight=16mm,
     bottom=30mm,
     top=15mm,
     bottom=20mm,
    includeheadfoot
]{geometry}

\usepackage[
	per-mode=symbol,
	separate-uncertainty=true,
]{siunitx}

\usepackage{booktabs}
\usepackage{caption}
\captionsetup[table]{skip=1ex}

\usepackage{tikz}
\usepackage{graphicx}
\graphicspath{{figures/}}

\usepackage{pgfgantt}

\usepackage{cleveref}

\usepackage{fancyhdr}
\fancypagestyle{unicamp}{
\renewcommand{\headrule}{}
\renewcommand{\footrule}{}
\fancyhead{}
\fancyfoot{}
\fancyhead[C]{\sffamily%
{\bfseries\fontsize{15.5pt}{1em}\selectfont\uppercase{Haute École Bruxelles-Brabant}}\\
\fontsize{11.3pt}{1.2em}\selectfont\uppercase{Faculty of Sciences}\\
\uppercase{Master of Cybersecurity \\ 2025-2026}}\\
\fancyfoot[C]{\sffamily\fontsize{9pt}{1em}\selectfont%
Rue Royale 67, 1000 Bruxelles, Belgium\\
    }
}

\pagestyle{plain}

\usepackage{setspace}

\usepackage{listings}  % For code listings
\usepackage{xcolor}    % For colors (if not already included)

\definecolor{codegreen}{rgb}{0,0.6,0}
\definecolor{codegray}{rgb}{0.5,0.5,0.5}
\definecolor{codepurple}{rgb}{0.58,0,0.82}
\definecolor{codeblue}{rgb}{0,0,1}
\definecolor{codebg}{rgb}{0.95,0.95,0.95}

\lstset{
  language=sh,                  % Change from C to sh (for Bash/shell commands)
  basicstyle=\ttfamily\small,   % Monospace font, small size
  keywordstyle=\color{codeblue}\bfseries,  % Keywords in bold blue
  stringstyle=\color{codepurple},          % Strings in purple
  commentstyle=\color{codegreen}\itshape,  % Comments in italic green
  identifierstyle=\color{black},           % Identifiers in black
  morekeywords={if,then,else,fi,for,do,done,while,until,case,esac,break,continue,return,exit,echo,sudo,chmod,gcc,python,angr,pwntools},  % Add Bash-specific keywords (extend as needed for angr/bash commands)
  sensitive=false,              % Case insensitive
  numbers=left,                 % Line numbers on left
  numberstyle=\tiny\color{codegray},  % Small gray line numbers
  stepnumber=1,                 % Number every line
  numbersep=5pt,                % Spacing from code
  backgroundcolor=\color{codebg},  % Light gray background
  showspaces=false,             % No visible spaces
  showstringspaces=false,       % No visible string spaces
  showtabs=false,               % No visible tabs
  frame=single,                 % Single frame around code
  rulecolor=\color{black},      % Black frame
  tabsize=2,                    % Tab size
  captionpos=b,                 % Caption below
  breaklines=true,              % Auto break long lines
  breakatwhitespace=false,      % Break anywhere
  escapeinside={\%*}{*)},       % Allow LaTeX escapes inside code
}
\begin{document}

\thispagestyle{unicamp}

\begin{center}

\null\vfill

{\scshape\large Secure Software Development \& Web Security\par}
{\scshape\large Computer Project\par}

\vskip 3\baselineskip

{\LARGE\bfseries Medical Records System\par}

\vskip 3\baselineskip

Candidates\\[1.5ex]
{\large\bfseries\begin{tabular}{@{}r@{\hspace{1.5em}}l@{}}
    Peetroons Simon & (519237) \\
    Andrianirina Mino & (604350) \\ 
    Botton David & (615056) \\
    Gerday Léandre & (616822) \\
    Varga Ferenc & (617441) \\
\end{tabular}}
\vskip 3\baselineskip

Advisor\\[1ex]
{\large\bfseries Professor Absil\par}

\end{center}

\vfill

\newpage


\pagenumbering{arabic}
\pagestyle{fancy}
\fancyhf{}
\renewcommand{\headrulewidth}{0.4pt}
\fancyfoot[C]{\thepage}
\clearpage
\begingroup
    \hypersetup{linkcolor=black}
    \tableofcontents
\endgroup
\clearpage
\pagenumbering{arabic}
\pagestyle{fancy}
\fancyhead[L]{\sffamily\small Secure System Design }
\fancyhead[R]{\sffamily\small 2026 Computer Project}
\fancyfoot[C]{\thepage}
\newpage

% Per Mélot Section 1 (Introduction): Start with a concise overview of the problem, your approach, and the report's structure. Avoid verbosity; focus on putting your work in value (mettre en valeur). Question: Does this section precisely state objectives without unnecessary background? Is it self-contained yet brief?

\section{Introduction}
The goal of this project is to implement a secure client/server system handling medical records, with an untrusted server and emphasis on security.

We designed our system as a Dockerized application using Django for the backend, React for the frontend, Nginx as a reverse proxy, and Step-CA for PKI management. Authentication relies on WebAuthn with PRF extensions for key derivation, supporting multi-device access via a primary-secondary hierarchy. All sensitive data, including medical records, is encrypted client-side using AES-GCM, with keys managed via ECDH for sharing between patients and appointed doctors. The server stores only ciphertexts and metadata, ensuring no access to plaintext even if compromised.

This approach prioritizes efficiency and security: WebAuthn over passwords to resist phishing, client-side E2EE to maintain confidentiality, and a local PKI chain (Root CA $\to$ Intermediate $\to$ Leaf certs) for TLS.

The report is structured as follows: Section 2 describes the system architecture; Section 3 details features and security measures; Section 4 lists all process flows; Section 5 evaluates threats; and Section 6 concludes. An appendix addresses the security checklist.

% Per Mélot Section 4.2 (Content): Ensure content is rigorous, scientific, and justified. Use tables/figures for clarity (e.g., architecture diagram). Question: Are claims backed by evidence/references? Did you avoid speculation?

\section{System Characteristics and Architecture}
The architecture is client/server, with clients driven by users (patients or doctors) and an untrusted server.

We containerized the system using Docker Compose, including services for the frontend (React with Tanstack, PNPM, Tailwind, Vite), backend (Django with custom WebAuthn), Nginx reverse proxy, and Step-CA for PKI. This ensures reproducibility and isolation, with Nginx handling TLS termination for client-server communication.



\subsection{The Server}
The server is not trusted. In particular, we do not know its set of public keys. Consequently, we provide a mechanism to "securely" transfer it and check its ownership via a local PKI chain: Root CA signs Intermediate CA, which issues leaf certificates for services (e.g., server.healthsecure.local). Fingerprints are computed and verified client-side.

No sensitive data is stored in plaintext; all medical records are ciphertexts. Metadata (e.g., request time, size, privileges, tree depth) is sent separately for anomaly detection, without depending on content.


\subsection{Users and Medical Records}
Users are patients or doctors. Doctors are trusted, with x509 certificates signed by the PKI for integrity and non-repudiation of attributes (e.g., organization).

Patients have first/last name, DOB, medical record (encrypted directory of dated files), and appointed doctors.

Doctors have first/last name and medical organization.

Medical records are client-encrypted directories; content/names/dates are sensitive and never exposed server-side.

% Per Mélot Section 4.1 (Structure): Use logical subsections per feature; ensure flow guides reader. Question: Is the structure balanced and hierarchical? Does each section build on the previous?

\section{Features and Security Implementation}
In each protocol, data exchange and storage are secured with emphasis on confidentiality, integrity, and additional measures like anti-injection.

We implemented more than basic protections, reCAPTCHA, and logging with separation of duties, favoring efficiency (e.g., WebAuthn over MFA for simplicity and security).

\subsection{User Registration, Authentication, and Revocation}
Registration generates WebAuthn credentials client-side, with PRF extension for KEK derivation. Doctors include x509 certs signed by PKI for attribute verification.

Authentication uses WebAuthn assertions, validating sign counts to detect clones. Sessions are short-lived (20 min) and encrypted.

Multi-device: Primary device (first registered) approves secondaries via one-time codes (<10 min expiry). Only primary can change credentials or revoke devices.

Revocation flushes sessions and deletes credentials, logged with metadata.


\emph{Avoided:} Traditional passwords (vulnerable to phishing/dictionary attacks). \emph{Preferred:} WebAuthn for hardware-bound, phishing-resistant auth.

Addresses Q2 (hardened auth: zero-knowledge WebAuthn), Q5 (non-repudiation: signatures), Q14 (auth not broken: per OWASP).

\subsection{Adding/Deleting a Doctor}

Patients add doctors via ECDH-shared secrets for DEK sharing; requires patient approval if doctor-initiated.
Deletion revokes access by removing shared keys.
Integrity ensured via Ed25519 signatures over lists.
\emph{Avoided:} Server-mediated sharing (compromises confidentiality). \emph{Preferred:} Client-side ECDH for E2EE sharing.
Addresses Q4 (sequence integrity: signatures), Q9 (request forgery: CSRF tokens).

\subsection{Viewing a Medical Record}
Patients/doctors decrypt records client-side using KEK/DEK. Server provides ciphertexts only to authorized users (via session checks).

No one else accesses; RBAC enforced backend.

Addresses Q1 (confidentiality: E2EE), Q3 (integrity: Ed25519 signatures).

\subsection{Uploading, Editing, and Deleting Files}
Files encrypted client-side with random DEK, signed with Ed25519. DEK encrypted with KEK (patient) and ECDH (doctors).

Doctor actions require patient approval.

Deletion overwrites with random data to mitigate remanence.

Addresses Q8 (remanence: overwrite), Q7 (injections: Django validators).

\subsection{Monitoring and Additional Security}
Logs record activity

% Per Mélot Section 4.2: Evaluate your work critically. Question: Did you assess limitations and efficiencies? Is this section concise yet comprehensive?

\section{Process Flows}

\subsection{Register}
\subsection{Login}

\subsection{etc..}



\section{Threat Modeling and Evaluation}
We performed threat modeling using STRIDE to identify risks and mitigations.

- \textbf{Spoofing}: Mitigated by WebAuthn; PKI prevents fake servers.
- \textbf{Tampering}: Ed25519 signatures on records/lists.
- \textbf{Repudiation}: Logs with non-repudiable signatures.
- \textbf{Information Disclosure}: E2EE; no plaintext on server.
- \textbf{Denial of Service}: reCAPTCHA, rate limiting.
- \textbf{Elevation of Privilege}: Primary-only management; RBAC.

Limitations: Incomplete logging separation; no ML anomaly detection due to data constraints. Efficient choices: WebAuthn reduces overhead vs. MFA; client-side crypto offloads server.

\printbibliography

% Per Mélot Annexes: Use for supplementary material. Question: Is this non-essential? Does it support main content without overwhelming?

\appendix
\section{Security Checklist Responses}
% \begin{table}[H]
% \centering
% \begin{tabular}{ll}
% \toprule
% Question & Response \\
% \midrule
% 1. Do I properly ensure confidentiality? & Yes, via client-side E2EE with AES-GCM; no plaintext transmitted/stored. Admin has no key access. \\
% 2. Did I harden my authentication scheme? & Yes, WebAuthn with PRF and zero-knowledge proofs; no Captcha/MFA needed due to built-in resistance. \\
% 3. Do I properly ensure integrity of stored data? & Yes, Ed25519 signatures on ciphertexts. \\
% 4. Do I properly ensure the integrity of sequences of items? & Yes, hash chains/signatures on doctor lists/records; tampering detected via verification. \\
% 5. Do I properly ensure non-repudiation? & Yes, Ed25519 signatures and PKI certs for actions/attributes. \\
% 6. Do my security features rely on secrecy, beyond cryptographic keys and access codes? & No, all rely on standards; no obscurity. \\
% 7. Am I vulnerable to injection? & No, Django sanitizes URL/SQL/JS inputs. \\
% 8. Am I vulnerable to data remanence attacks? & No, deletions overwrite with random data. \\
% 9. Am I vulnerable to fraudulent request forgery? & No, CSRF tokens. \\
% 10. Am I monitoring enough user activity...? & Yes, AuthenticationLog with sanitization; anomaly detection via metadata; no whistleblower yet. \\
% 11. Am I using components with known vulnerabilities? & No, deps updated via PNPM. \\
% 12. Is my system updated? & Yes, Docker images pinned to latest secure versions. \\
% 13. Is my access control broken (cf. OWASP 10)? & No, RBAC with primary hierarchy. \\
% 14. Is my authentication broken (cf. OWASP 10)? & No, WebAuthn aligns with OWASP. \\
% 15. Are my general security features misconfigured (cf. OWASP 10)? & No, TLS, secure cookies configured. \\
% \bottomrule
% \end{tabular}
% \caption{Security Checklist Answers}
% \end{table}

\section{Additional Material}
% Code snippets, diagrams, setup instructions (e.g., Docker).


\end{document}
